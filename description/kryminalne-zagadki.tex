\documentclass[a4paper,12pt]{article}

\usepackage[utf8]{inputenc}
\usepackage[T1]{fontenc}
\usepackage{graphicx}
\usepackage{listings}
\usepackage{xcolor} 
\usepackage{booktabs}
\usepackage{geometry}

\geometry{a4paper, margin=1in}
\lstdefinelanguage{Cypher}{
  keywords={MATCH, RETURN, CREATE, DELETE, WHERE, SET, MERGE, LIMIT},
  sensitive=true,
  comment=[l]{//},
  morecomment=[s]{/*}{*/},
  morestring=[b]",
  morestring=[b]'
}
% Ustawienie wyglądu tytułu
\title{\vspace{2cm} \textbf{\Huge Kryminalne zagadki} \\[1cm] \large Uniwersytet Gdański \\ Grafowe Bazy Danych \vspace{2cm}}

% Autorzy i data
\author{\textsc{Michał Redkwa, Maciej Marzec}}
\date{12.01.2025}

\begin{document}

% Strona tytulowa
\maketitle

\newpage

\section{Wstęp}
\subsection{Założenia projektu}

W ramach projektu zaliczeniowego stworzyliśmy grafową bazę danych, której celem jest pokazanie, w jaki sposób za pomocą algorytmów i zapytań można zidentyfikować osobę odpowiedzialną za popełnienie przestępstwa. Na bazie zostały przeprowadzone zapytania, które pozwalają na wskazanie osób, które mogą być bezpośrednio związane z dowodami przestępstwa lub pełnić rolę potencjalnych świadków. Dzięki zastosowaniu zaawansowanych algorytmów, jesteśmy w stanie ocenić ich znaczenie w kontekście śledztwa oraz wytypować osoby o najmocniejszych powiązaniach z podejrzanymi.

\subsection{Opis wierzchołków oraz relacji pomiędzy nimi}

\begin{center}
    \textbf{\textit{Person}}
\end{center}

Wierzchołek \textbf{Person} reprezentuje osobę, posiada on właściwości: imię, nazwisko, wiek, długość oraz kolor włosów, kolor oczu, numer telefonu, numer konta bankowego oraz zawód jaki wykonuje. Jest on powiązany z następującymi wierzchołkami:

\begin{itemize}
    \item \textbf{Street} – relacja LIVES\_ON pomiędzy Person a Street.
    \item \textbf{Shop} – relacja BOUGHT\_AT lub VISITED pomiędzy Person a Shop. Relacja posiada dodatkowo właściwości, takie jak: typ wizyty, ilośc zakupów, ilośc wizyt.
    \item \textbf{Institution} – TODO ?? Zrobić powiązanie z bankiem jak ktoś ma tam konto?
\end{itemize}

\begin{center}
    \textbf{\textit{Street}}
\end{center}

Wierzchołek \textbf{Street} reprezentuje ulicę, zawiera takie własciowiści jak: nazwa, miasto, kod, typ. Ulice sa powiązane ze sobą za pomocą relacji CONNECTED\_TO.

\begin{center}
    \textbf{\textit{Institution}}
\end{center}

Wierzchołek \textbf{Institution} reprezentuje instytucję, jego własciowościami jest nazwa oraz jaki jest to typ instytucji. Wierzchołek ten powiązany jest z ulicami za pomocą relacji LOCATED\_AT.

\begin{center}
    \textbf{\textit{Shop}}
\end{center}

Wierzchołek \textbf{Shop} reprezentuje sklep. Posiada następujące własciwości: nazwa, typ oraz godziny otwarcia. Jest powiązany z ulicą za pomocą relacji LOCATED\_AT.

\begin{center}
    \textbf{\textit{Crime}}
\end{center}

Wierzchołek \textbf{Crime} reprezentuje zbrodnie. Posiada własciwości: typ przestępstwa, data, wyrok. Jest powiązany z daną ulicą, sklepem czy instytucją za pomocą relacji o nazwie COMMITTED\_AT.

\begin{center}
    \textbf{\textit{Evidence}}
\end{center}

Wierzchołek \textbf{Evidence} reprezentuje dowód, który posiada własciwości takie jak: typ, opis oraz date złożenia lub jego znalezienia. Jest powiązany z wierzchołkiem zbrodnia za pomocą relacji EVIDENCE\_IN.

\section{Zapytania Cypher}

W ramach demonstracji przeszukiwania grafowej bazy danych naszym podejrzanym w przypadku włamania do sklepu będzie Paula Harris. Na początek wypiszmy wszystkie informacje jakie mamy o tej osobie.

\begin{center}
\begin{minipage}{0.8\linewidth}
\begin{lstlisting}[language=Cypher, basicstyle=\small, breaklines=true]
MATCH (p:Person {firstName: 'Paula', lastName: 'Harris'})-[r]->(related)
RETURN p, r, related
\end{lstlisting}
\end{minipage}
\end{center}

\begin{figure}[h!]
    \centering
    \includegraphics[width=0.7\textwidth]{paula_harris.png}
\end{figure}

Teraz prześledźmy wszystkie dowody związane ze sprawą włamania i spróbujmy na podstawie ich znaleźć osobę, która własnie jemu odpowiada.

\begin{center}
\begin{minipage}{0.8\linewidth}
\begin{lstlisting}[language=Cypher, basicstyle=\small, breaklines=true]
MATCH (n)-[r]->(m)
WHERE n.crimeType = 'Robbery' AND n.date = '2023-05-12' 
RETURN r, n, m;
\end{lstlisting}
\end{minipage}
\end{center}

\begin{figure}[h!]
    \centering
    \includegraphics[width=0.7\textwidth]{robbery.png} 
\end{figure}

\newpage
Na podstawie zeznań świadków:

\begin{itemize}
    \item A witness reported seeing a person with medium-length red hair leaving the scene of the robbery at around 12:05 AM.
    \item A witness reported that the suspect had a height between 160 and 170 cm.
    \item A witness saw a person with red hair near the scene of the robbery at approximately 11:50 PM.
\end{itemize}

oraz nagrania z kamery, które wg opisu mówi nam: CCTV footage shows a person with medium-length red hair and blue eyes near the robbery location at 11:55 PM.
\newline
\newline Wyszukajmy osoby pasujące do tego rysopisu:

\begin{center}
\begin{minipage}{0.8\linewidth}
\begin{lstlisting}[language=Cypher, basicstyle=\small, breaklines=true]
MATCH (person:Person)
WHERE person.hairColor = 'red' AND person.height >= 160 AND person.height <= 170 AND person.eyeColor = 'blue'
RETURN person.firstName, person.lastName, person.height, person.hairColor, person.eyeColor
LIMIT 5;
\end{lstlisting}
\end{minipage}
\end{center}

\begin{table}[h!]
\centering
\begin{tabular}{|l|l|c|c|c|}
\hline
\textbf{First Name} & \textbf{Last Name} & \textbf{Height (cm)} & \textbf{Hair Color} & \textbf{Eye Color} \\ \hline
Daisy               & Anderson           & 160                 & red                 & blue               \\ \hline
Paula               & Harris             & 167                 & red                 & blue               \\ \hline
Wanda               & Holmes             & 170                 & red                 & blue               \\ \hline
Grace               & Reed               & 164                 & red                 & blue               \\ \hline
Maya                & Xavier             & 163                 & red                 & blue               \\ \hline
\end{tabular}
\caption{Tabela przedstawiająca dane osób z kolorem włosów "red" i kolorem oczu "blue".}
\label{tab:person_data_with_eyes}
\end{table}

Jak widać, mamy na naszej liście osobę, która popełniła to przestępstwo, jednak nie daje nam to ostatecznego potwierdzenia, że jest to osoba, która powinna zostać uznana za winną popełnionego przestępstwa. Został jeszcze jeden dowód z centrali telefonicznej w obrębie obszaru, gdzie popełniono przestępstwo.
\newline
\newline
"A phone with the number which ends on ...951 was found to have been frequently used near the crime scene, with activity under the alias "AngryVoice" reported around the time of the robbery."

\begin{center}
\begin{minipage}{0.8\linewidth}
\begin{lstlisting}[language=Cypher, basicstyle=\small, breaklines=true]
MATCH (person:Person)
WHERE toString(person.phoneNumber) ENDS WITH '951'
RETURN person.firstName, person.lastName, person.phoneNumber, person.hairColor, person.eyeColor
LIMIT 5;
\end{lstlisting}
\end{minipage}
\end{center}
\newpage
\begin{table}[h!]
\centering
\begin{tabular}{|l|l|l|l|l|}
\hline
\textbf{First Name} & \textbf{Last Name} & \textbf{Phone Number} & \textbf{Hair Color} & \textbf{Eye Color} \\ \hline
Paula              & Harris             & 753486951             & red                 & blue               \\ \hline
Henry              & Clark              & 852147951             & grey                & brown              \\ \hline
Emma               & Parker             & 852147951             & blonde              & green              \\ \hline
Brian              & Mills              & 456789951             & brown               & blue               \\ \hline
David              & Owens              & 753456951             & black               & brown              \\ \hline
\end{tabular}
\caption{Tabela przedstawiająca dane osobowe, numery telefonów oraz kolory włosów i oczu.}
\label{tab:person_data}
\end{table}

Jak widać, osoba którą podejrzewaliśmy znajduje się na tej liście - sprawa została rozwiązana.

\section{Algorytmy}

\end{document}