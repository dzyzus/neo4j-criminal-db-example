\documentclass[a4paper,12pt]{article}

\usepackage[utf8]{inputenc}
\usepackage[T1]{fontenc}

% Ustawienie wyglądu tytułu
\title{\vspace{2cm} \textbf{\Huge Kryminalne zagadki} \\[1cm] \large Uniwersytet Gdański \\ Grafowe Bazy Danych \vspace{2cm}}

% Autorzy i data
\author{\textsc{Michał Redkwa, Maciej Marzec}}
\date{12.01.2025}

\begin{document}

% Strona tytulowa
\maketitle

\newpage

\section{Wstęp}
\subsection{Założenia projektu}

W ramach projektu zaliczeniowego stworzyliśmy grafową bazę danych, której celem jest pokazanie, w jaki sposób za pomocą algorytmów i zapytań można zidentyfikować osobę odpowiedzialną za popełnienie przestępstwa. Na bazie zostały przeprowadzone zapytania, które pozwalają na wskazanie osób, które mogą być bezpośrednio związane z dowodami przestępstwa lub pełnić rolę potencjalnych świadków. Dzięki zastosowaniu zaawansowanych algorytmów, jesteśmy w stanie ocenić ich znaczenie w kontekście śledztwa oraz wytypować osoby o najmocniejszych powiązaniach z podejrzanymi.

\subsection{Opis wierzchołków oraz relacji pomiędzy nimi}

\begin{center}
    \textbf{\textit{Person}}
\end{center}

Wierzchołek \textbf{Person} reprezentuje osobę, posiada on właściwości: imię, nazwisko, wiek, długość oraz kolor włosów, kolor oczu, numer telefonu, numer konta bankowego oraz zawód jaki wykonuje. Jest on powiązany z następującymi wierzchołkami:

\begin{itemize}
    \item \textbf{Street} – relacja LIVES\_ON pomiędzy Person a Street.
    \item \textbf{Shop} – relacja BOUGHT\_AT lub VISITED pomiędzy Person a Shop. Relacja posiada dodatkowo właściwości, takie jak: typ wizyty, ilośc zakupów, ilośc wizyt.
    \item \textbf{Institution} – TODO ?? Zrobić powiązanie z bankiem jak ktoś ma tam konto?
\end{itemize}

\begin{center}
    \textbf{\textit{Street}}
\end{center}

Wierzchołek \textbf{Street} reprezentuje ulicę, zawiera takie własciowiści jak: nazwa, miasto, kod, typ. Ulice sa powiązane ze sobą za pomocą relacji CONNECTED\_TO.

\begin{center}
    \textbf{\textit{Institution}}
\end{center}

Wierzchołek \textbf{Institution} reprezentuje instytucję, jego własciowościami jest nazwa oraz jaki jest to typ instytucji. Wierzchołek ten powiązany jest z ulicami za pomocą relacji LOCATED\_AT.

\begin{center}
    \textbf{\textit{Shop}}
\end{center}

Wierzchołek \textbf{Shop} reprezentuje sklep. Posiada następujące własciwości: nazwa, typ oraz godziny otwarcia. Jest powiązany z ulicą za pomocą relacji LOCATED\_AT.

\begin{center}
    \textbf{\textit{Crime}}
\end{center}

Wierzchołek \textbf{Crime} reprezentuje zbrodnie. Posiada własciwości: typ przestępstwa, data, wyrok. Jest powiązany z daną ulicą, sklepem czy instytucją za pomocą relacji o nazwie COMMITTED\_AT.

\begin{center}
    \textbf{\textit{Evidence}}
\end{center}

Wierzchołek \textbf{Evidence} reprezentuje dowód, który posiada własciwości takie jak: typ, opis oraz date złożenia lub jego znalezienia. Jest powiązany z wierzchołkiem zbrodnia za pomocą relacji EVIDENCE\_IN.

\section{Zapytania}

\section{Algorytmy}

\end{document}